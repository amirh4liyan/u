\documentclass[12pt]{article}
    \usepackage[breakable]{tcolorbox}
    \usepackage{minted}
    \usepackage{parskip} % Stop auto-indenting (to mimic markdown behaviour)

    % \renewcommand{\familydefault}{\sfdefault}

    % Basic figure setup, for now with no caption control since it's done
    % automatically by Pandoc (which extracts ![](path) syntax from Markdown).
    \usepackage{graphicx}
    % Maintain compatibility with old templates. Remove in nbconvert 6.0
    \let\Oldincludegraphics\includegraphics
    % Ensure that by default, figures have no caption (until we provide a
    % proper Figure object with a Caption API and a way to capture that
    % in the conversion process - todo).
    \usepackage{caption}
    \DeclareCaptionFormat{nocaption}{}
    \captionsetup{format=nocaption,aboveskip=0pt,belowskip=0pt}

    \usepackage{float}
    \floatplacement{figure}{H} % forces figures to be placed at the correct location
    \usepackage{xcolor} % Allow colors to be defined
    \usepackage{enumerate} % Needed for markdown enumerations to work
    \usepackage{geometry} % Used to adjust the document margins
    \usepackage{textcomp} % defines textquotesingle
    % Hack from http://tex.stackexchange.com/a/47451/13684:
    \AtBeginDocument{%
        \def\PYZsq{\textquotesingle}% Upright quotes in Pygmentized code
    }
    \usepackage{upquote} % Upright quotes for verbatim code
    \usepackage{eurosym} % defines \euro

    \usepackage{iftex}
    \ifPDFTeX
        \usepackage[T1]{fontenc}
        \IfFileExists{alphabeta.sty}{
              \usepackage{alphabeta}
          }{
              \usepackage[mathletters]{ucs}
              \usepackage[utf8x]{inputenc}
          }
    \else
        \usepackage{fontspec}
        \usepackage{unicode-math}
    \fi

    \usepackage{fancyvrb} % verbatim replacement that allows latex
    \usepackage{grffile} % extends the file name processing of package graphics
                         % to support a larger range
    \makeatletter % fix for old versions of grffile with XeLaTeX
    \@ifpackagelater{grffile}{2019/11/01}
    {
      % Do nothing on new versions
    }
    {
      \def\Gread@@xetex#1{%
        \IfFileExists{"\Gin@base".bb}%
        {\Gread@eps{\Gin@base.bb}}%
        {\Gread@@xetex@aux#1}%
      }
    }
    \makeatother
    \usepackage[Export]{adjustbox} % Used to constrain images to a maximum size
    \adjustboxset{max size={0.9\linewidth}{0.9\paperheight}}

    % The hyperref package gives us a pdf with properly built
    % internal navigation ('pdf bookmarks' for the table of contents,
    % internal cross-reference links, web links for URLs, etc.)
    \usepackage{hyperref}
    % The default LaTeX title has an obnoxious amount of whitespace. By default,
    % titling removes some of it. It also provides customization options.
    \usepackage{titling}
    \usepackage{longtable} % longtable support required by pandoc >1.10
    \usepackage{booktabs}  % table support for pandoc > 1.12.2
    \usepackage{array}     % table support for pandoc >= 2.11.3
    \usepackage{calc}      % table minipage width calculation for pandoc >= 2.11.1
    \usepackage[inline]{enumitem} % IRkernel/repr support (it uses the enumerate* environment)
    \usepackage[normalem]{ulem} % ulem is needed to support strikethroughs (\sout)
                                % normalem makes italics be italics, not underlines
    \usepackage{soul}      % strikethrough (\st) support for pandoc >= 3.0.0
    \usepackage{mathrsfs}


    % Colors 
    
    % Colors for the hyperref package
    \definecolor{urlcolor}{rgb}{0,.145,.698}
    \definecolor{linkcolor}{rgb}{.71,0.21,0.01}
    \definecolor{citecolor}{rgb}{.12,.54,.11}

    % ANSI colors
    \definecolor{ansi-black}{HTML}{3E424D}
    \definecolor{ansi-black-intense}{HTML}{282C36}
    \definecolor{ansi-red}{HTML}{E75C58}
    \definecolor{ansi-red-intense}{HTML}{B22B31}
    \definecolor{ansi-green}{HTML}{00A250}
    \definecolor{ansi-green-intense}{HTML}{007427}
    \definecolor{ansi-yellow}{HTML}{DDB62B}
    \definecolor{ansi-yellow-intense}{HTML}{B27D12}
    \definecolor{ansi-blue}{HTML}{208FFB}
    \definecolor{ansi-blue-intense}{HTML}{0065CA}
    \definecolor{ansi-magenta}{HTML}{D160C4}
    \definecolor{ansi-magenta-intense}{HTML}{A03196}
    \definecolor{ansi-cyan}{HTML}{60C6C8}
    \definecolor{ansi-cyan-intense}{HTML}{258F8F}
    \definecolor{ansi-white}{HTML}{C5C1B4}
    \definecolor{ansi-white-intense}{HTML}{A1A6B2}
    \definecolor{ansi-default-inverse-fg}{HTML}{FFFFFF}
    \definecolor{ansi-default-inverse-bg}{HTML}{000000}

    % My Favorite Inverted Colors
    \definecolor{inverted1}{HTML}{37BBBF}
    \definecolor{inverted2}{HTML}{D28F4B}
    \definecolor{inverted3}{HTML}{C673B9}
    \definecolor{inverted4}{HTML}{0581E5}
    \definecolor{inverted5}{HTML}{9FBD59}
    \definecolor{inverted6}{HTML}{FFDA4D}

    % common color for the border for error outputs.
    \definecolor{outerrorbackground}{HTML}{FFDFDF}

    % commands and environments needed by pandoc snippets
    % extracted from the output of `pandoc -s`
    \providecommand{\tightlist}{%
      \setlength{\itemsep}{0pt}\setlength{\parskip}{0pt}}
    \DefineVerbatimEnvironment{Highlighting}{Verbatim}{commandchars=\\\{\}}
    % Add ',fontsize=\small' for more characters per line
    \newenvironment{Shaded}{}{}
    \newcommand{\KeywordTok}[1]{\textcolor[rgb]{0.00,0.44,0.13}{\textbf{{#1}}}}
    \newcommand{\DataTypeTok}[1]{\textcolor[rgb]{0.56,0.13,0.00}{{#1}}}
    \newcommand{\DecValTok}[1]{\textcolor[rgb]{0.25,0.63,0.44}{{#1}}}
    \newcommand{\BaseNTok}[1]{\textcolor[rgb]{0.25,0.63,0.44}{{#1}}}
    \newcommand{\FloatTok}[1]{\textcolor[rgb]{0.25,0.63,0.44}{{#1}}}
    \newcommand{\CharTok}[1]{\textcolor[rgb]{0.25,0.44,0.63}{{#1}}}
    \newcommand{\StringTok}[1]{\textcolor[rgb]{0.25,0.44,0.63}{{#1}}}
    \newcommand{\CommentTok}[1]{\textcolor[rgb]{0.38,0.63,0.69}{\textit{{#1}}}}
    \newcommand{\OtherTok}[1]{\textcolor[rgb]{0.00,0.44,0.13}{{#1}}}
    \newcommand{\AlertTok}[1]{\textcolor[rgb]{1.00,0.00,0.00}{\textbf{{#1}}}}
    \newcommand{\FunctionTok}[1]{\textcolor[rgb]{0.02,0.16,0.49}{{#1}}}
    \newcommand{\RegionMarkerTok}[1]{{#1}}
    \newcommand{\ErrorTok}[1]{\textcolor[rgb]{1.00,0.00,0.00}{\textbf{{#1}}}}
    \newcommand{\NormalTok}[1]{{#1}}

    % Additional commands for more recent versions of Pandoc
    \newcommand{\ConstantTok}[1]{\textcolor[rgb]{0.53,0.00,0.00}{{#1}}}
    \newcommand{\SpecialCharTok}[1]{\textcolor[rgb]{0.25,0.44,0.63}{{#1}}}
    \newcommand{\VerbatimStringTok}[1]{\textcolor[rgb]{0.25,0.44,0.63}{{#1}}}
    \newcommand{\SpecialStringTok}[1]{\textcolor[rgb]{0.73,0.40,0.53}{{#1}}}
    \newcommand{\ImportTok}[1]{{#1}}
    \newcommand{\DocumentationTok}[1]{\textcolor[rgb]{0.73,0.13,0.13}{\textit{{#1}}}}
    \newcommand{\AnnotationTok}[1]{\textcolor[rgb]{0.38,0.63,0.69}{\textbf{\textit{{#1}}}}}
    \newcommand{\CommentVarTok}[1]{\textcolor[rgb]{0.38,0.63,0.69}{\textbf{\textit{{#1}}}}}
    \newcommand{\VariableTok}[1]{\textcolor[rgb]{0.10,0.09,0.49}{{#1}}}
    \newcommand{\ControlFlowTok}[1]{\textcolor[rgb]{0.00,0.44,0.13}{\textbf{{#1}}}}
    \newcommand{\OperatorTok}[1]{\textcolor[rgb]{0.40,0.40,0.40}{{#1}}}
    \newcommand{\BuiltInTok}[1]{{#1}}
    \newcommand{\ExtensionTok}[1]{{#1}}
    \newcommand{\PreprocessorTok}[1]{\textcolor[rgb]{0.74,0.48,0.00}{{#1}}}
    \newcommand{\AttributeTok}[1]{\textcolor[rgb]{0.49,0.56,0.16}{{#1}}}
    \newcommand{\InformationTok}[1]{\textcolor[rgb]{0.38,0.63,0.69}{\textbf{\textit{{#1}}}}}
    \newcommand{\WarningTok}[1]{\textcolor[rgb]{0.38,0.63,0.69}{\textbf{\textit{{#1}}}}}


    % Define a nice break command that doesn't care if a line doesn't already
    % exist.
    \def\br{\hspace*{\fill} \\* }
    % Math Jax compatibility definitions
    \def\gt{>}
    \def\lt{<}
    \let\Oldtex\TeX
    \let\Oldlatex\LaTeX
    \renewcommand{\TeX}{\textrm{\Oldtex}}
    \renewcommand{\LaTeX}{\textrm{\Oldlatex}}


    % Document parameters
    % Document title
    \title{Answers of Homework 2}
    
    % Document author
    \author{Amirhossein Alian} 

    % Pygments definitions
    \makeatletter
    \def\PY@reset{\let\PY@it=\relax \let\PY@bf=\relax%
        \let\PY@ul=\relax \let\PY@tc=\relax%
        \let\PY@bc=\relax \let\PY@ff=\relax}
    \def\PY@tok#1{\csname PY@tok@#1\endcsname}
    \def\PY@toks#1+{\ifx\relax#1\empty\else%
        \PY@tok{#1}\expandafter\PY@toks\fi}
    \def\PY@do#1{\PY@bc{\PY@tc{\PY@ul{%
        \PY@it{\PY@bf{\PY@ff{#1}}}}}}}
    \def\PY#1#2{\PY@reset\PY@toks#1+\relax+\PY@do{#2}}


    \@namedef{PY@tok@w}{\def\PY@tc##1{\textcolor[RGB]{153, 0, 25}{##1}}}
    \@namedef{PY@tok@c}{\def\PY@tc##1{\textcolor[RGB]{221, 17, 68}{##1}}}
    \@namedef{PY@tok@e}{\def\PY@tc##1{\textcolor[RGB]{153, 153, 136}{##1}}}
    \@namedef{PY@tok@m}{\def\PY@tc##1{\textcolor[RGB]{0, 153, 153}{##1}}}
    \@namedef{PY@tok@k}{\def\PY@tc##1{\textcolor[RGB]{64, 64, 64}{##1}}}
    \@namedef{PY@tok@ft}{\def\PY@tc##1{\textcolor[RGB]{68, 85, 136}{##1}}}
    \@namedef{PY@tok@fn}{\def\PY@tc##1{\textcolor[RGB]{153, 0, 0}{##1}}}

    % For linebreaks inside Verbatim environment from package fancyvrb.
    \makeatletter
        \newbox\Wrappedcontinuationbox
        \newbox\Wrappedvisiblespacebox
        \newcommand*\Wrappedvisiblespace {\textcolor{red}{\textvisiblespace}}
        \newcommand*\Wrappedcontinuationsymbol {\textcolor{red}{\llap{\tiny$\m@th\hookrightarrow$}}}
        \newcommand*\Wrappedcontinuationindent {3ex }
        \newcommand*\Wrappedafterbreak {\kern\Wrappedcontinuationindent\copy\Wrappedcontinuationbox}
        % Take advantage of the already applied Pygments mark-up to insert
        % potential linebreaks for TeX processing.
        %        {, <, #, %, $, ' and ": go to next line.
        %        _, }, ^, &, >, - and ~: stay at end of broken line.
        % Use of \textquotesingle for straight quote.
        \newcommand*\Wrappedbreaksatspecials {%
            \def\PYGZus{\discretionary{\char`\_}{\Wrappedafterbreak}{\char`\_}}%
            \def\PYGZob{\discretionary{}{\Wrappedafterbreak\char`\{}{\char`\{}}%
            \def\PYGZcb{\discretionary{\char`\}}{\Wrappedafterbreak}{\char`\}}}%
            \def\PYGZca{\discretionary{\char`\^}{\Wrappedafterbreak}{\char`\^}}%
            \def\PYGZam{\discretionary{\char`\&}{\Wrappedafterbreak}{\char`\&}}%
            \def\PYGZlt{\discretionary{}{\Wrappedafterbreak\char`\<}{\char`\<}}%
            \def\PYGZgt{\discretionary{\char`\>}{\Wrappedafterbreak}{\char`\>}}%
            \def\PYGZsh{\discretionary{}{\Wrappedafterbreak\char`\#}{\char`\#}}%
            \def\PYGZpc{\discretionary{}{\Wrappedafterbreak\char`\%}{\char`\%}}%
            \def\PYGZdl{\discretionary{}{\Wrappedafterbreak\char`\$}{\char`\$}}%
            \def\PYGZhy{\discretionary{\char`\-}{\Wrappedafterbreak}{\char`\-}}%
            \def\PYGZsq{\discretionary{}{\Wrappedafterbreak\textquotesingle}{\textquotesingle}}%
            \def\PYGZdq{\discretionary{}{\Wrappedafterbreak\char`\"}{\char`\"}}%
            \def\PYGZti{\discretionary{\char`\~}{\Wrappedafterbreak}{\char`\~}}%
        }
        % Some characters . , ; ? ! / are not pygmentized.
        % This macro makes them "active" and they will insert potential linebreaks
        \newcommand*\Wrappedbreaksatpunct {%
            \lccode`\~`\.\lowercase{\def~}{\discretionary{\hbox{\char`\.}}{\Wrappedafterbreak}{\hbox{\char`\.}}}%
            \lccode`\~`\,\lowercase{\def~}{\discretionary{\hbox{\char`\,}}{\Wrappedafterbreak}{\hbox{\char`\,}}}%
            \lccode`\~`\;\lowercase{\def~}{\discretionary{\hbox{\char`\;}}{\Wrappedafterbreak}{\hbox{\char`\;}}}%
            \lccode`\~`\:\lowercase{\def~}{\discretionary{\hbox{\char`\:}}{\Wrappedafterbreak}{\hbox{\char`\:}}}%
            \lccode`\~`\?\lowercase{\def~}{\discretionary{\hbox{\char`\?}}{\Wrappedafterbreak}{\hbox{\char`\?}}}%
            \lccode`\~`\!\lowercase{\def~}{\discretionary{\hbox{\char`\!}}{\Wrappedafterbreak}{\hbox{\char`\!}}}%
            \lccode`\~`\/\lowercase{\def~}{\discretionary{\hbox{\char`\/}}{\Wrappedafterbreak}{\hbox{\char`\/}}}%
            \catcode`\.\active
            \catcode`\,\active
            \catcode`\;\active
            \catcode`\:\active
            \catcode`\?\active
            \catcode`\!\active
            \catcode`\/\active
            \lccode`\~`\~
        }
    \makeatother


    \let\OriginalVerbatim=\Verbatim
    \makeatletter
    \renewcommand{\Verbatim}[1][1]{%
        %\parskip\z@skip
        \sbox\Wrappedcontinuationbox {\Wrappedcontinuationsymbol}%
        \sbox\Wrappedvisiblespacebox {\FV@SetupFont\Wrappedvisiblespace}%
        \def\FancyVerbFormatLine ##1{\hsize\linewidth
            \vtop{\raggedright\hyphenpenalty\z@\exhyphenpenalty\z@
                \doublehyphendemerits\z@\finalhyphendemerits\z@
                \strut ##1\strut}%
        }%
        % If the linebreak is at a space, the latter will be displayed as visible
        % space at end of first line, and a continuation symbol starts next line.
        % Stretch/shrink are however usually zero for typewriter font.
        \def\FV@Space {%
            \nobreak\hskip\z@ plus\fontdimen3\font minus\fontdimen4\font
            \discretionary{\copy\Wrappedvisiblespacebox}{\Wrappedafterbreak}
            {\kern\fontdimen2\font}%
        }%

        % Allow breaks at special characters using \PYG... macros.
        \Wrappedbreaksatspecials
        % Breaks at punctuation characters . , ; ? ! and / need catcode=\active
        \OriginalVerbatim[#1,codes*=\Wrappedbreaksatpunct]%
    }
    \makeatother


    % Exact colors from NB
    \definecolor{incolor}{HTML}{303F9F}
    \definecolor{outcolor}{HTML}{D84315}
    \definecolor{cellborder}{HTML}{161616}
    \definecolor{mycellborder}{HTML}{303030}
    \definecolor{cellbackground}{HTML}{FFFFFF}
    \definecolor{mycellbackground}{HTML}{3F3F3F}
    \definecolor{mydarkcellbackground}{HTML}{2B2B2B}
    

    \definecolor{mybackground}{HTML}{000000}
    \definecolor{mytextcolor}{HTML}{FFFFFF}


    % prompt
    \makeatletter
    \newcommand{\boxspacing}{\kern\kvtcb@left@rule\kern\kvtcb@boxsep}
    \makeatother
    \newcommand{\prompt}[4]{
        {\ttfamily\llap{{\color{#2}[#3]:\hspace{3pt}#4}}\vspace{-\baselineskip}}
    }


    % Prevent overflowing lines due to hard-to-break entities
    \sloppy
    % Setup hyperref package
    \hypersetup{
      breaklinks=true,  % so long urls are correctly broken across lines
      colorlinks=true,
      urlcolor=urlcolor,
      linkcolor=linkcolor,
      citecolor=citecolor,
      }
    % Slightly bigger margins than the latex defaults
    
    \geometry{verbose,tmargin=1in,bmargin=1in,lmargin=1in,rmargin=1in}

 
        \begin{document}

    \maketitle
    
	\section{First Case}
	
	\subsection{Code}
    \begin{tcolorbox}[breakable, size=fbox, boxrule=1pt, pad at break*=1mm,colback=cellbackground, colframe=cellborder]
\begin{minted}{c++}
// in the name of God
// Amirhossein Alian
// Computer Engineering
// 4021120017
// Date: 1402-09-22

#include <iostream>

using namespace std;

int main()
{
        int a, b, c, d;
        int s, k;

        a = 2;
        b = ++a * 2;
        c = a > b - 5 ? b * 3 : a + b;
        d = a + b + c - 2;
        s = (k = 2, k = k * 2, ++k);

        cout << "[out]: s = " << s;
        cout << ", k = " << k;
        cout << ", a = " << a;
        cout << ", b = " << b;
        cout << ", c = " << c;
        cout << ", d = " << d << endl;
        return 0;
}
\end{minted}
\end{tcolorbox}

	\subsection{Output}
    \begin{tcolorbox}[breakable, size=fbox, boxrule=1pt, pad at break*=1mm,colback=cellbackground, colframe=cellborder]

\begin{Verbatim}[commandchars=\\\{\}]
[out]: s = 5, k = 5, a = 3, b = 6, c = 18, d = 25
\end{Verbatim}
\end{tcolorbox}
\pagebreak

        \section{Second Case}
	
	\subsection{Code}
    \begin{tcolorbox}[breakable, size=fbox, boxrule=1pt, pad at break*=1mm, colback=cellbackground, colframe=cellborder]
\begin{minted}{c++}
// in the name of God
// Amirhossein Alian
// Computer Engineering
// 4021120017
// Date: 1402-09-22

#include <iostream>

using namespace std;

int main()
{
        int a = 5, b = 8, temp;

        cout << "[out]: i think this exam is simple.";

        if (a > b || a + 3 > b - 2)
                temp = a * b + a;
        else
                temp = a * b - a;
        cout << "\n[out]: temp = " << temp << endl;
        return 0;
}
\end{minted}
\end{tcolorbox}

	\subsection{Output}
    \begin{tcolorbox}[breakable, size=fbox, boxrule=1pt, pad at break*=1mm, colback=cellbackground, colframe=cellborder]
\begin{Verbatim}[commandchars=\\\{\}]
[out]: i think this exam is simple.
[out]: temp = 45
\end{Verbatim}
\end{tcolorbox}

\pagebreak
	\section{Third Case}
	
	\subsection{Code}
   \begin{tcolorbox}[breakable, size=fbox, boxrule=1pt, pad at break*=1mm, colback=cellbackground, colframe=cellborder]
\begin{minted}{c++}
// in the name of God
// Amirhossein Alian
// Computer Engineering
// 4021120017
// Date: 1402-09-22

#include <iostream>

using namespace std;

int main()
{
        double x;
        cout << "[in-double] Enter x: ";
        cin >> x;

        double f;
        if (x < 0)
                f = (-4)*(x*x*x) + 2*(x*x) + 5;
        else if (x > 0)
                f = (x*x) - 3*(x) - 2;
        else
                f = -3;
        cout << "[out]: f = " << f << endl;
        return 0;
}
\end{minted}
\end{tcolorbox}

	\subsection{Input and Output}

    \begin{tcolorbox}[breakable, size=fbox, boxrule=1pt, pad at break*=1mm, colback=cellbackground!5!white, colframe=gray!75!black, title=Test Case 1]
\begin{Verbatim}[commandchars=\\\{\}]
[in-double] Enter x: 5
[out]: f = 8
\end{Verbatim}
\end{tcolorbox}


    \begin{tcolorbox}[breakable, size=fbox, boxrule=1pt, pad at break*=1mm, colback=cellbackground!5!white, colframe=gray!75!black, title=Test Case 2]
\begin{Verbatim}[commandchars=\\\{\}]
[in-double] Enter x: 0
[out]: f = -3
\end{Verbatim}
\end{tcolorbox}


    \begin{tcolorbox}[breakable, size=fbox, boxrule=1pt, pad at break*=1mm, colback=cellbackground!5!white, colframe=gray!75!black, title=Test Case 3]
\begin{Verbatim}[commandchars=\\\{\}]
[in-double] Enter x: -1
[out]: f = 11
\end{Verbatim}
\end{tcolorbox}


    \begin{tcolorbox}[breakable, size=fbox, boxrule=1pt, pad at break*=1mm, colback=cellbackground!5!white, colframe=gray!75!black, title=Test Case 4]
\begin{Verbatim}[commandchars=\\\{\}]
[in-double] Enter x: -0.1
[out]: f = 5.024
\end{Verbatim}
\end{tcolorbox}

\pagebreak

        \section{Fourth Case}
        \subsection{Code}
   \begin{tcolorbox}[breakable, size=fbox, boxrule=1pt, pad at break*=1mm, colback=cellbackground, colframe=cellborder]
\begin{minted}{c++}
// in the name of God
// Amirhossein Alian
// Computer Engineering
// 4021120017
// Date: 1402-09-22

#include <iostream>

using namespace std;

int main()
{
        double a, b;
        cout << "[in-double] Enter A: ";
        cin >> a;
        cout << "[in-double] Enter B: ";
        cin >> b;

        if (a < 0)
                a *= -1;
        if (b < 0)
                b *= -1;

        int n = a + b - 2 * (a * b);
        cout << "[out] N: " << n << endl;
        return 0;
}

\end{minted}
\end{tcolorbox}

	\subsection{Input and Output}

    \begin{tcolorbox}[breakable, size=fbox, boxrule=1pt, pad at break*=1mm, colback=cellbackground!5!white, colframe=gray!75!black, title=Test Case 1]
\begin{Verbatim}[commandchars=\\\{\}]
[in-double] Enter A: 7
[in-double] Enter B: 4
[out] N: -45
\end{Verbatim}
\end{tcolorbox}


    \begin{tcolorbox}[breakable, size=fbox, boxrule=1pt, pad at break*=1mm, colback=cellbackground!5!white, colframe=gray!75!black, title=Test Case 2]
\begin{Verbatim}[commandchars=\\\{\}]
[in-double] Enter A: -7 
[in-double] Enter B: 4
[out] N: -45
\end{Verbatim}
\end{tcolorbox}


    \begin{tcolorbox}[breakable, size=fbox, boxrule=1pt, pad at break*=1mm, colback=cellbackground!5!white, colframe=gray!75!black, title=Test Case 3]
\begin{Verbatim}[commandchars=\\\{\}]
[in-double] Enter A: -7
[in-double] Enter B: -4
[out] N: -45
\end{Verbatim}
\end{tcolorbox}

    \begin{tcolorbox}[breakable, size=fbox, boxrule=1pt, pad at break*=1mm, colback=cellbackground!5!white, colframe=gray!75!black, title=Test Case 3]
\begin{Verbatim}[commandchars=\\\{\}]
[in-double] Enter A: 7
[in-double] Enter B: -4
[out] N: -45
\end{Verbatim}
\end{tcolorbox}

\pagebreak

  \section{Fifth Case}

        \subsection{First Code}
   \begin{tcolorbox}[breakable, size=fbox, boxrule=1pt, pad at break*=1mm, colback=cellbackground, colframe=cellborder]
\begin{minted}{c++}
// Amirhossein Alian
// Computer Engineering
// 4021120017
// Date: 1402-09-22

#include <iostream>

using namespace std;

int main()
{
        bool isFound = false;
        int target = 13900904;
        for (int i = 0; i < 20; i++) {
                int n;
                cout << "[in-int]: Guess the Target: ";
                cin >> n;
                if (n == target) {
                        cout << "[out]: Win\n";
                        isFound = true;
                        break;
                }
        }

        if (! isFound)
                cout << "[out]: Game Over!\n";
        return 0;
}
\end{minted}
\end{tcolorbox}

 	\subsection{Explanations}
    \begin{tcolorbox}[breakable, size=fbox, boxrule=1pt, pad at break*=1mm, colback=cellbackground, colframe=cellborder]
\begin{Verbatim}[commandchars=\\\{\}]
This code is a game that gives you 20 chances to guess the number
\end{Verbatim}
\end{tcolorbox}

        \subsection{Second Code}
   \begin{tcolorbox}[breakable, size=fbox, boxrule=1pt, pad at break*=1mm, colback=cellbackground, colframe=cellborder]
\begin{minted}{c++}
// Amirhossein Alian
// Computer Engineering
// 4021120017
// Date: 1402-09-22

#include <iostream>

using namespace std;

int main()
{
        double a, b;
        char op;
        
        cout << "1) +\n";
        cout << "2) -\n";
        cout << "3) *\n";
        cout << "4) /\n";
        cout << "5) > (max)\n";
        cout << "6) < (min)\n\n";
        cout << "Enter a code: ";
        cin >> op;

        cout << "[in-double] Enter A: ";
        cin >> a;
        cout << "[in-double] Enter B: ";
        cin >> b;

        switch (op) {
        case '1':
                cout << a + b;
                break;
        case '2':
                cout << a - b;
                break;
        case '3':
                cout << a * b;
                break;
        case '4':
                if (b == 0)
                        cout << "[Error] Division by Zero";
                else
                        cout << a / b;
                break;
        case '5':
                double max; max = (a > b) ? a : b; 
                cout << max;
                break;
        case '6':
                double min; min = (a < b) ? a : b; 
                cout << min;
                break;
        default:
                cout << "Invalid Operator!";
                break;
        } 
        cout << endl;
        return 0;
}
\end{minted}
\end{tcolorbox}

	\subsection{Input and Output}

    \begin{tcolorbox}[breakable, size=fbox, boxrule=1pt, pad at break*=1mm, colback=cellbackground!5!white, colframe=gray!75!black, title=Test Case 1]
\begin{Verbatim}[commandchars=\\\{\}]
[out]: 1) +
[out]: 2) -
[out]: 3) *
[out]: 4) /
[out]: 5) > (max)
[out]: 6) < (min)

[in-int] Enter a code: 3
[in-double] Enter A: 11.11
[in-double] Enter B: 2.54
[out]: 28.2194
\end{Verbatim}
\end{tcolorbox}

    \begin{tcolorbox}[breakable, size=fbox, boxrule=1pt, pad at break*=1mm, colback=cellbackground!5!white, colframe=gray!75!black, title=Test Case 2]
\begin{Verbatim}[commandchars=\\\{\}]
[out]: 1) +
[out]: 2) -
[out]: 3) *
[out]: 4) /
[out]: 5) > (max)
[out]: 6) < (min)

[in-int] Enter a code: 4
[in-double] Enter A: 9
[in-double] Enter B: 2
[out]: 4.5
\end{Verbatim}
\end{tcolorbox}

    \begin{tcolorbox}[breakable, size=fbox, boxrule=1pt, pad at break*=1mm, colback=cellbackground!5!white, colframe=gray!75!black, title=Test Case 3]
\begin{Verbatim}[commandchars=\\\{\}]
[out]: 1) +
[out]: 2) -
[out]: 3) *
[out]: 4) /
[out]: 5) > (max)
[out]: 6) < (min)

[in-int] Enter a code: 5
[in-double] Enter A: 14.3
[in-double] Enter B: 17.6
[out]: 17.6
\end{Verbatim}
\end{tcolorbox}

\pagebreak

  \section{Sixth Case}
        \subsection{First Code}
   \begin{tcolorbox}[breakable, size=fbox, boxrule=1pt, pad at break*=1mm, colback=cellbackground, colframe=cellborder]
\begin{minted}{c++}
// Amirhossein Alian
// Computer Engineering
// 4021120017
// Date: 1402-09-22

#include <iostream>

using namespace std;

int main()
{
        int n;
        cout << "[in-int] Enter N: ";
        cin >> n;

        double x = 0;
        for (double i = 2; i <= n+1; i += 2)
                x +=  1 / (i * (i + 1));
        
        if (n % 2 == 1)
                x += 1 / (double) (n+2);
        
        cout << "[out]: " << x << endl;
        return 0;
}
\end{minted}
\end{tcolorbox}

 	\subsection{Explanations}
    \begin{tcolorbox}[breakable, size=fbox, boxrule=1pt, pad at break*=1mm, colback=cellbackground, colframe=cellborder]
\begin{Verbatim}[commandchars=\\\{\}]
This code receives a number N and calculate N'th sequence of (1/2 - 1/3 + 1/4 - 1/5 + 1/6 - 1/7 + ...)
\end{Verbatim}
\end{tcolorbox}

        \subsection{Second Code}
   \begin{tcolorbox}[breakable, size=fbox, boxrule=1pt, pad at break*=1mm, colback=cellbackground, colframe=cellborder]
\begin{minted}{c++}
// Amirhossein Alian
// Computer Engineering
// 4021120017
// Date: 1402-09-22

#include <iostream>

using namespace std;

int main()
{
        int n;
        cout << "[in-int] Enter N: ";
        cin >> n;

        int counter = 0;
        int end = sqrt(n);
        for (int i = 1; i <= end; i++)
                if (n % i == 0)
                        counter++;
        
        counter *= 2;
        
        if (sqrt(n) == end)
                counter--;

        cout << "[out]: " << counter << endl;
        return 0;
}
\end{minted}
\end{tcolorbox}

 	\subsection{Explanations}
    \begin{tcolorbox}[breakable, size=fbox, boxrule=1pt, pad at break*=1mm, colback=cellbackground, colframe=cellborder]
\begin{Verbatim}[commandchars=\\\{\}]
This code receives a number and counts its divisors.
\end{Verbatim}
\end{tcolorbox}

\pagebreak

  \section{Seventh Case}
        \subsection{Code}
   \begin{tcolorbox}[breakable, size=fbox, boxrule=1pt, pad at break*=1mm, colback=cellbackground, colframe=cellborder]
\begin{minted}{c++}
// in the name of God
// Amirhossein Alian
// Computer Engineering
// 4021120017
// Date: 1402-09-22

#include <iostream>

using namespace std;

int main()
{
        int st, i, x = 0;
        cout << "[in-int] Enter Student no only three low digits: ";
        cin >> st;

        i = 1;
        while(i <= 3) {
                x += st % 10;
                st /= 10;
                i++;
        }

        cout << "[out]: X = " << x;
        cout << "[out]: \n -----------";
        return 0;
}
\end{minted}
\end{tcolorbox}

	\subsection{Input and Output}

    \begin{tcolorbox}[breakable, size=fbox, boxrule=1pt, pad at break*=1mm, colback=cellbackground!5!white, colframe=gray!75!black, title=Test Case 1]
\begin{Verbatim}[commandchars=\\\{\}]
[in-int] Enter Student no only three low digits: 17
[out]: X = 8
[out]: -----------
\end{Verbatim}
\end{tcolorbox}

    % Add a bibliography block to the postdoc
    
\end{document}
