% Optimized XePersian Lab Report Template
% Lightweight, clean, and efficient for bilingual reports with Persian/English text and images

\documentclass[a4paper,12pt]{article}

% --- Essential Packages ---
\usepackage[margin=1in]{geometry}
\usepackage{graphicx}      % for images
\usepackage{caption}       % better captions
\usepackage{subcaption}    % subfigures
\usepackage{float}         % control figure placement
\usepackage{booktabs}      % for tables
\usepackage{hyperref}      % clickable links
\usepackage{xcolor}        % color support
\usepackage{fancyhdr}      % headers and footers
\usepackage[breakable]{tcolorbox}
\usepackage{minted}
\usepackage{xepersian}     % Persian/English text support

% Exact colors from NB
    \definecolor{incolor}{HTML}{303F9F}
    \definecolor{outcolor}{HTML}{D84315}
    \definecolor{cellborder}{HTML}{161616}
    \definecolor{mycellborder}{HTML}{303030}
    \definecolor{cellbackground}{HTML}{FFFFFF}
    \definecolor{mycellbackground}{HTML}{3F3F3F}
    \definecolor{mydarkcellbackground}{HTML}{2B2B2B}
    

    \definecolor{mybackground}{HTML}{000000}
    \definecolor{mytextcolor}{HTML}{FFFFFF}

% --- Font Setup ---
\settextfont{IRANSans.ttf}
\settextfont[BoldFont={IRANSans_Bold.ttf}]{IRANSans.ttf}
\setlatintextfont{Times New Roman}
\setlatintextfont[BoldFont={Times New Roman Bold}]{Times New Roman}

% --- Page Style ---
\pagestyle{fancy}
\fancyhf{}
\rhead{\thepage}
\lhead{\lr{CN Homeworks}}
\renewcommand{\headrulewidth}{0.4pt}

% --- Basic Formatting ---
\setlength{\parskip}{0.6em}
\setlength{\parindent}{0pt}
\renewcommand{\figurename}{شکل}
\renewcommand{\tablename}{جدول}
\definecolor{engblue}{RGB}{0, 80, 180}
\definecolor{engred}{RGB}{255, 0, 0}
\definecolor{enggreen}{RGB}{0, 180, 80}
\definecolor{red}{RGB}{180, 0, 0}
\newcommand{\engb}[1]{\textcolor{engblue}{\lr{#1}}}
\newcommand{\engr}[1]{\textcolor{engred}{\lr{#1}}}

% --- Title Page ---
\title{\textbf{\lr{Computer Network Extra Homework 1}}}  % Lab Report Title
\author{امیرحسین عالیان \\\lr{4021120017}}
\date{\today}

\begin{document}

\maketitle

\vspace{1cm}
\tableofcontents
\newpage

% ----------------------------
%   Sections
% ----------------------------
\section{تمرین امتیازی اول}
\subsection{بررسی پارامترهای مختلف دستور tracert با آدرس IP یا Hostname دلخواه}
\subsubsection{کاربرد دستور}
دستور \engb{tracert} (در ویندوز) یا معادل پیشرفته آن \engb{traceroute} (در لینوکس و مک) یکی از ابزارهای پایه در شبکه و عیب‌یابی ارتباطات اینترنتی است.
این دستور نشان می‌دهد که بسته‌ها (\engb{packets}) برای رسیدن به یک مقصد (مثلاً یک وب‌سایت) از چه مسیر و چه روترهایی عبور می‌کنند.

\subsubsection{مفهوم کلی}

وقتی ما پکت‌ها را به مقصدی می‌فرستیم (مثلاً \engr{google.com})، آن‌ها از چندین روتر میانی عبور می‌کنند تا به مقصد برسند. \engb{tracert} تعداد این \engb{“hop”‌} ها را یکی‌یکی افزایش می‌دهد و پاسخ هر روتر را ثبت می‌کند.

در واقع:\\
در گام اول \engr{TTL=1}، روتر اول پاسخ می‌دهد.\\
در گام دوم \engr{TTL=2}، روتر دوم پاسخ می‌دهد.\\

و همین‌طور ادامه دارد تا به مقصد برسد یا حداکثر تعداد \engb{hop} (معمولاً 30) تمام شود.

\subsubsection{سوییچ های دستور}
\vspace*{0.5cm}
\begin{latin}
\begin{LTR}
\begin{table}[h!]
	\centering
	\begin{tabular}{cll}
	\toprule
	\textbf{Options} & \textbf{Parameter} & \textbf{Explanation} \\
	\midrule
\textbf{-d} & & \textbf{Do not resolve addresses to hostnames.} \\
\textbf{-h} & \textbf{maximum-hops} & \textbf{Maximum number of hops to search for target.} \\
-j & host-list & Loose source route along host-list (IPv4-only). \\
\textbf{-w} & \textbf{timeout} & \textbf{Wait timeout milliseconds for each reply.} \\
-R & & Trace round-trip path (IPv6-only). \\
-S & srcaddr & Source address to use (IPv6-only). \\
-4 & & Force using IPv4. \\ 
-6 & & Force using IPv6. \\
	\bottomrule
	\end{tabular}
	\caption{\rl{لیست سوییچ های قابل استفاده}}
	\label{tab:components}
\end{table}
\end{LTR}
\end{latin}
\vspace*{0.65cm}

\pagebreak
\subsection{نتیجه اجرای دستور به ازای چند سوییچ مختلف}
\subsubsection{اجرای بدون سوییچ}
\begin{tcolorbox}[breakable, size=fbox, boxrule=1pt, pad at break*=1mm,colback=cellbackground, colframe=cellborder]
\begin{minted}[formatcom=\setLTR]{powershell}
C:\Users\amir>tracert www.google.com

Tracing route to www.google.com [2a00:1450:4001:813::2004]
over a maximum of 30 hops:

  1   151 ms   120 ms   120 ms  2a09:bac6::
  2   188 ms   137 ms   126 ms  2400:cb00:470:1000::1
  3   135 ms   133 ms   133 ms  2400:cb00:71:2::2
  4   123 ms   118 ms   119 ms  2400:cb00:71:200::7
  5   104 ms   113 ms   115 ms  2a00:1450:8463:c0::1
  6   122 ms   134 ms     *     2001:4860:0:1::69a8
  7   122 ms   138 ms   126 ms  2001:4860:0:1::40b3
  8   201 ms   147 ms   138 ms  2a00:1450:4001:813::2004

Trace complete.
\end{minted}
\end{tcolorbox}
\subsubsection{اجرای با سوییچ -d}
\begin{tcolorbox}[breakable, size=fbox, boxrule=1pt, pad at break*=1mm,colback=cellbackground, colframe=cellborder]
\begin{minted}[formatcom=\setLTR]{powershell}
C:\Users\amir>tracert -d www.google.com

Tracing route to www.google.com [2a00:1450:4001:813::2004]
over a maximum of 30 hops:

  1   245 ms   186 ms   224 ms  2a09:bac6::
  2   177 ms   148 ms   146 ms  2400:cb00:470:1000::1
  3   137 ms   164 ms   171 ms  2400:cb00:71:2::2
  4   152 ms   138 ms   123 ms  2400:cb00:71:200::7
  5   139 ms   133 ms   141 ms  2a00:1450:8463:c0::1
  6   133 ms   117 ms   155 ms  2001:4860:0:1::69a8
  7   122 ms   111 ms   122 ms  2001:4860:0:1::40b3
  8   170 ms   144 ms   123 ms  2a00:1450:4001:813::2004

Trace complete.
\end{minted}
\end{tcolorbox}
\subsubsection{اجرای با سوییچ -h}
\begin{tcolorbox}[breakable, size=fbox, boxrule=1pt, pad at break*=1mm,colback=cellbackground, colframe=cellborder]
\begin{minted}[breaklines, formatcom=\setLTR]{powershell}
C:\Users\amir>tracert -h 10 google.com

Tracing route to google.com [2a00:1450:4001:830::200e]
over a maximum of 10 hops:

  1   189 ms   147 ms   142 ms  2a09:bac6::
  2   120 ms   150 ms   148 ms  2400:cb00:470:1000::1
  3   138 ms   115 ms   142 ms  2400:cb00:71:2::1
  4   137 ms   120 ms   138 ms  2a00:1450:8153::1
  5   946 ms     *      224 ms  2a00:1450:8153::1
  6     *        *        *     Request timed out.
  7     *      263 ms   309 ms  2001:4860:0:1::589b
  8     *      231 ms   222 ms  fra24s11-in-x0e.1e100.net [2a00:1450:4001:830::200e]

Trace complete.
\end{minted}
\end{tcolorbox}

\subsubsection{سایت دانشگاه شریف}
\begin{tcolorbox}[breakable, size=fbox, boxrule=1pt, pad at break*=1mm,colback=cellbackground, colframe=cellborder]
\begin{minted}[breaklines, formatcom=\setLTR]{powershell}
C:\Users\amir>tracert -w 100 www.sharif.ir

Tracing route to www.sharif.ir [152.89.13.27]
over a maximum of 30 hops:

  1   270 ms     *      298 ms  104.28.0.0
  2     *        *      416 ms  8.18.50.1
  3     *      441 ms     *     162.158.84.110
  4   237 ms     *      160 ms  BE15-362.br03.nyc06.as3491.net [63.218.233.177]
  5   280 ms     *      261 ms  63.222.114.193
  6     *      361 ms     *     185.233.140.66
  7     *        *        *     Request timed out.
  8     *        *        *     Request timed out.
  9     *        *        *     Request timed out.
 10     *        *        *     Request timed out.
 11     *        *        *     Request timed out.
 12     *        *        *     Request timed out.
 13   433 ms     *      310 ms  152.89.13.27

Trace complete.
\end{minted}
\end{tcolorbox}


\end{document}
